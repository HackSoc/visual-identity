\documentclass[10pt,a4paper,oneside]{article}
\pagestyle{headings}

\usepackage{graphicx}
\usepackage[margin=0.9in, top=0.7in]{geometry}
\usepackage{ucs}
\usepackage{multirow}
\usepackage{tabularx}
\usepackage{fontspec}
\usepackage{adjustbox}
\usepackage{titlesec}
\usepackage{changepage}
\usepackage{pdflscape}
\usepackage{multicol}
\usepackage{minted}
\usepackage{hyperref}

\setmonofont{Consolas}

\author{Michael Walker}
\title{HackSoc - the computer science society}

\pagestyle{empty}
\setmainfont{DejaVu Sans}

\setlength{\parindent}{0cm}
\setlength{\parskip}{1em}

\titleformat*{\section}{\huge\fontspec{Sanchez Regular}}
\titleformat*{\subsection}{\Large\fontspec{Sanchez Regular}}

\titlespacing{\section}{0pt}{\parskip}{-\parskip}
\titlespacing{\subsection}{0pt}{\parskip}{-\parskip}

\newcolumntype{L}{>{\raggedright\arraybackslash\small}X}

\newcommand{\event}[2]{\subsection*{#1 \hfill \textbf{\fontspec{DejaVu
        Sans} \footnotesize #2}}}

\begin{document}

% %%%%% HackSoc Header

\begin{adjustbox}{minipage=0.9\linewidth,center}
  {\fontsize{2.75cm}{1em} \fontspec{Rokkitt} HackSoc}

  \vspace{0.35cm}

  \hfill {\fontsize{0.9cm}{1em} \fontspec{Sanchez Regular} the computer science society}
\end{adjustbox}

% %%%%% Icon Square

\vspace{0.3cm}

\begin{tabular}{llllllll}
\multicolumn{1}{m{1cm}}{\includegraphics[width=1cm, height=1cm]{www-512.png}} &
hacksoc.org &
\multicolumn{1}{m{1cm}}{\includegraphics[width=1cm, height=1cm]{irc-512.png}} &
freenode \#cs-york &
\multicolumn{1}{m{1cm}}{\includegraphics[width=1cm, height=1cm]{facebook-512.png}} &
/groups/hacksoc &
\multicolumn{1}{m{1cm}}{\includegraphics[width=1cm, height=1cm]{twitter-512.png}} &
@HackSoc
\end{tabular}

% %%%%% Welcome Letter

\vspace{0.2cm}

\section*{Welcome to York}

\textbf{HackSoc} is the student-run society for Computer Science
students. We're open to all members of the university, but we're more
likely to be something \textit{you're} interested in, so welcome,
potential member!

We run a lot of events throughout the term, and are always trying out
new things. We're excited to be introducing study groups this term,
and a functional programming course in Spring term. We also have talks
every three weeks, two of which will be given by invited industrial
speakers next term, and at least two socials a week. All of our events
are open to anyone, so feel free to come along to anything which takes
your fancy.

Why not get to know us before term starts? We're on Facebook, Twitter,
and IRC. If you've never used IRC before, we have a little guide at
\textbf{hacksoc.org/irc.html}

See you next term!

\vspace{-1cm}

\begin{flushright}
  {\Large — Michael Walker, HackSoc Chair}\\
  {\small \textbf{barrucadu} on irc}
\end{flushright}

\vspace{0.6cm}

% %%%%% Highlights

\section*{What's on?}

\event{Talks}{Weeks 3, 6, and 9}

We have three invited speakers a term: two industrial and one
academic. In addition to that, we also run lightning talk sessions
where anyone can get up and present for five minutes.

\vspace{0.1cm}

\event{Study \& Reading Groups}{Fortnightly}

Interested in cryptography? Computer vision? Distributed systems? Or
maybe you just want to chat about some classic CS papers over
biscuits. This term we're introducing study groups. We haven't decided
exactly what we'll run yet, so keep an eye on the social media and
website.

\vspace{0.1cm}

\event{Science Week}{Week 8}

A large gathering of science societies is taking place next
term\ldots{} a week-long series of talks and events, open to all,
showcasing the different sciences available at York and how they are
all connected. See \url{YorkScienceWeek.uk} for more information.

\vspace{0.1cm}

\subsection*{Regular Socials}

We also have a lot of socials: pub trips every other Tuesday; every
\textit{other} other Friday we're playing boardgames and eating cake;
and every Wednesday we're having lunch in the Glasshouse (Langwith
College bar). We also have occasional movie and video game nights.

\pagebreak

\begin{landscape}
\begin{multicols}{3}

\section*{Python 2 Cheatsheet}

\renewcommand{\arraystretch}{1.5}

\subsection*{Basics}

\mintinline{python}|42|\\
{\small The integer 42}

\mintinline{python}|42.0|\\
{\small The floating-point value 42}

\mintinline{python}|"hello world"|\\
{\small The string "hello world"}

\mintinline{python}|u"hello world"|\\
{\small  The unicode string "hello world"}

\mintinline{python}|foo = bar|\\
{\small \mintinline{python}|foo| takes on the value
  \mintinline{python}|bar|}

\mintinline{python}|import foo|\\
{\small Import the module \mintinline{python}|foo|}

\vspace{-0.25cm}
\subsection*{Input/Output}

\mintinline{python}|print foo|\\
Print the value of \mintinline{python}|foo|

\mintinline{python}|foo = input()|\\
Read in a string, evaluate it, and store the result in
\mintinline{python}|foo|. You probably shouldn't do this.

\mintinline{python}|foo = raw_input()|\\
Read in a string and store it in \mintinline{python}|foo|

\mintinline{python}|f = open("path")|\\
{\small Open the file at that path}

\mintinline{python}|f.close()|\\
{\small Close the open file}

\mintinline{python}|f.read()|\\
{\small Read the file contents into a string}

\vspace{-0.25cm}
\subsection*{Builtin Functions}

\begin{tabularx}{6cm}{ccc}
\mintinline{python}|abs()|&
\mintinline{python}|dir()|&
\mintinline{python}|len()|\\
\mintinline{python}|all()|&
\mintinline{python}|float()|&
\mintinline{python}|map()|\\
\mintinline{python}|any()|&
\mintinline{python}|filter()|&
\mintinline{python}|ord()|\\
\mintinline{python}|bool()|&
\texttt{\PYGdefault{n+nb}{help}\PYGdefault{p}{()}}&
\mintinline{python}|range()|\\
\mintinline{python}|chr()|&
\mintinline{python}|int()|&
\mintinline{python}|str()|
\end{tabularx}

\vspace{-0.25cm}
\subsection*{Control Flow}

\mintinline{python}|if ...: ... elif ...: ... else: ...|\\
{\small Execute the first block of code with a true condition}

\mintinline{python}|while ...: ...|\\
{\small Execute the code until the condition is false}

\mintinline{python}|for x in [...]: ...|\\
{\small Execute the code once for every value in the list}

\mintinline{python}|break| 
{\small Exit the current loop}

\mintinline{python}|continue| 
{\small Skip to the next iteration}

\vspace{-0.25cm}
\subsection*{Lists}

\begin{tabularx}{6cm}{lL}
\multicolumn{2}{l}{For \mintinline{python}|a = [0,1,2,3,4,5]|,}\\
%
\mintinline{python}|len(a)| &
6\\
%
\mintinline{python}|a[3]| &
3\\
%
\mintinline{python}|a[-1]| &
5\\
%
\mintinline{python}|a[1:]| &
\mintinline{python}|[1,2,3,4,5]|\\
%
\mintinline{python}|a[2:4]| &
\mintinline{python}|[2,3]|\\
%
\mintinline{python}|[x * 2 for x in a]| &
\mintinline{python}|[0,2,4,6,8,10]|\\
%
\mintinline{python}|[x for x in a if x < 3]| &
\mintinline{python}|[0,1,2]|
\end{tabularx}

\vspace{-0.25cm}
\subsection*{Dictionaries}

\begin{tabularx}{6cm}{lL}
\multicolumn{2}{l}{For \mintinline{python}|a = {"a": 1, "b": 2}|,}\\
%
\mintinline{python}|a.keys()| &
\mintinline{python}|["a", "b"]|\\
%
\mintinline{python}|a.values()| &
\mintinline{python}|[1, 2]|\\
%
\mintinline{python}|a.items()| &
\mintinline{python}|[("a", 1), ("b", 2)]|\\
%
\mintinline{python}|a["a"]| &
1
\end{tabularx}

\vspace{-0.25cm}
\subsection*{Functions}

\begin{minted}{python}
def foo(a, b="apples"):
    return "I have {} {}.".format(a, b)
\end{minted}
\begin{verbatim}
>>> foo(5)
I have 5 apples.
>>> foo(5, b="pears")
I have 5 pears.
\end{verbatim}

\begin{minted}{python}
def bar(*args, **kwargs):
    print str(args) + " and " + str(kwargs)
\end{minted}
\begin{verbatim}
>>> bar(1, 2, 3, hello="world")
(1, 2, 3) and {'hello': 'world'}
\end{verbatim}

\vspace{-0.25cm}
\subsection*{Classes \& Methods}

\begin{minted}{python}
class Foo(object):
    def __init__(self, x):
        self.x = x

    def xTimesTwo(self):
        return self.x * 2
\end{minted}
\begin{verbatim}
>>> foo = Foo(5)
>>> foo.xTimesTwo()
10
\end{verbatim}
\end{multicols}
\end{landscape}

\end{document}